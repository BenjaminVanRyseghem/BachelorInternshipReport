\paragraph{Presentation: } The goal of the RMoD team is to help re-modularization of object-oriented applications. This goal follows two complementary lines: re-engineering and definition of new constructors for programming languages. To help re-engineering, new analyses are proposed in order to understand and remodularize big applications (specialized metrics, adapted visualizations, \emph{etc}). In the context of programming languages, constructors for the modularity features and new systems modules validation are performed. The team is also working on a secured kernel for Pharo, an \gls{IDE} for Smalltalk used and maintained by the team.
%%NA: Ca donne l'impression que pharo est exclusivement un projet de l'equipe.
%%quelque chose comme "used by the team which is also an active member of its community"

\paragraph{Applications re-modularization: } The evolution of an application is limited by strong dependencies between its inner parts. That's why it's crucial to answer the following questions:
%%NA: ne pas utiliser " mais `` et ''
\emph{``How can we substitute a part by another one with minimal impact~?''}, \emph{``How to identify reusable elements~?''} or \emph{``How to modularize an application when there is wrong links~?''}. To answer those questions is the goal of Moose, the team software analysis environment, provides a set of analyses.
%%NA: de nouveau ce n'est pas l'environnement de l'equipe, c'est un environnement que nous utiliser et dont nous faisont parti de la comunauté ...
This work is divided in tree parts :
	\begin{itemize}
		\item Tools to understand big applications (packages/modules);
		\item Analysis for remodularization;
		\item Software quality.
	\end{itemize}
	
\paragraph{Semantics elements for modularity.} This second line focuses on the definition of new semantics elements for languages in order to construct flexible and reconfigurable software. The team continues its efforts on Traits and Classboxes but also works on new areas such as security in dynamic languages. It works on:
	\begin{itemize}
		\item The definition of a \emph{Traits-only} language and;
		\item Reconciliation between reflexive languages and security.
	\end{itemize}