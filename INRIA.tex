\paragraph{Presentation:} \gls{INRIA} is a french institution under the dual supervision of the ministries of research and industry which goal is to undertake research in basic and applied sciences and information technologies and communication (SITC). The institute also provides a strong technology transfer in a close attention to training through research, dissemination of scientific and technical development, expertise and participation in international programs.

\paragraph{Composition: } \gls{INRIA} accommodates 3800 people in its eight research centers located in Rocquencourt, Rennes, Sophia Antipolis, Grenoble, Nancy, Bordeaux, Lille and Saclay, 2800 of them are scientists from \gls{INRIA} and partner organizations (CNRS, universities, colleges) working in over 160 project teams of joint research. Many \gls{INRIA} researchers are also professors and their students (about 1000) are preparing their thesis within the project teams of \gls{INRIA} research.

\paragraph{The research center of Lille: } The \gls{INRIA}  Lille - Nord Europe, led by David Simplot-Rys, gathers from its inception 10 research teams located in a building of 4000m$^2$ acquired with the help of local government and European funds. It hosts more than 220 people, nearly half is paid by the Institute. This \gls{INRIA}  center is an asset for the competitiveness of Nord - Pas de Calais in research and innovation.
